\documentclass[Main.tex]{subfiles}
\renewcommand{\loigiai}[1]{%
	\par\noindent%
	{\color{dndo}\reflectbox{\Large\WritingHand}\ {\fmmfamily\LARGE Lời giải.}} #1
}
\begin{document}
	\newcommand{\nhanmanh}[2][darkmidnightblue]{{\large\color{#1}\textit{#2}}}
	\setcounter{tocdepth}{1}
	\setcounter{secnumdepth}{3}
	\begin{center}
		\begin{tcolorbox}[enhanced,hbox,
			left=8mm,right=8mm,boxrule=0.55pt,
			bottom=3pt,
			colback=cyan!5,colframe=darkmidnightblue,
			% drop fuzzy midday shadow=black!50!yellow,
			drop lifted shadow=black!50!darkmidnightblue,arc is angular,
			before=\par\vspace*{-1mm},after=\par\bigskip]
			{\Large\bfseries\sffamily\color{darkmidnightblue} MỤC LỤC}
		\end{tcolorbox}
	\end{center}
	
	\vspace*{-0.5cm}
	\makeatletter
	\@starttoc{toc}
	\makeatother
	
	\def\muctieu{Trong chương này, chúng ta tìm hiểu những nội dung sau: mệnh đề toán học, tập hợp và các phép toán trên tập hợp.}
	\chapter{Mệnh đề toán học tập hợp}
	
	\section{Mệnh đề toán học}
	\taoNdongke[5]{hd}
	\taoNdongke[5]{luyentap}
	\dongkevd
	\noindent
	H' Maryam: \lq\lq\textit{Số $15$ chia hết cho $5$}\rq\rq
	
	\noindent
	Phương: \lq\lq\textit{Việt Nam là một nước ở khu vực Đông Nam Á}\rq\rq
	
	\begin{cauhoikd}
		Trong hai phát biểu trên, phát biểu nào là mệnh đề toán học?
	\end{cauhoikd}
	
	\subsection{Mệnh đề toán học}
	
	\begin{hd}
		\begin{enumerate}
			\item Phát biểu của bạn H'Maryam có phải là một câu khẳng định về tính chất chia hết trong toán học hay không?
			\item Phát biểu của bạn Phương có phải là một câu khẳng định về một sự kiện trong toán học hay không?
		\end{enumerate}
		\loigiai{Phát biểu của bạn H'Maryam là một mệnh đề khẳng định về một sự kiện trong toán học, gọi là mệnh đề toán học.
			Như vậy, phát biểu của bạn Phương không phải là mệnh đề toán học.
		}
	\end{hd}
	\begin{luuy}
		Chú ý: Khi không sợ nhầm lẫn, ta thường gọi tắt mệnh đề toán học là mệnh đề.
	\end{luuy}
	
	%%%==============Vidu1==============%%%
	\begin{vd} Phát biểu nào sau đây là một mệnh đề toán học?
		\begin{enumerate}
			\item Số 7 là số nguyên tố
			\item Đi học đi!
			\item Paris là thủ đô của Đức.
			\item Nếu hôm nay trời mưa, tôi sẽ mang ô.
		\end{enumerate}
		\loigiai{
			\begin{enumerate}
				\item Đây một mệnh đề toán học.
				\item là một câu ra lệnh.
				\item là một câu khẳng định nhưng không phải là mệnh đề toán học.
				\item là một câu điều kiện không phải câu khẳng định
			\end{enumerate}
		}
	\end{vd}
	%%%==============HetVidu1==============%%%
	\begin{luyentap}
		Nêu hai ví dụ về mệnh đề
		\loigiai{}
	\end{luyentap}
	\begin{hd}
		Trong hai mệnh đề toán học sau đây, mệnh đề nào là một khẳng định đúng?
		Mệnh đề nào là một khẳng định sai?\\
		$P$: "Tổng hai góc đối của một tứ giác nội tiếp bằng $180^{\circ}$ ";\\
		$Q:$ " $\sqrt{2}$ là số hữu tì".
		\loigiai{}
	\end{hd}
	\begin{tomtat}
		Mỗi mệnh đề toán học phải hoặc đúng hoặc sai. Một mệnh đề toán học không thể vừa đúng, vừa sai.
	\end{tomtat}
\newpage
%%%==============Vidu1==============%%%
	\begin{paracol}{2}
		\begin{vd}
			Tìm mệnh đề đúng trong những mệnh đề sau: \\
			$A$: "Tam giác có ba cạnh"; \\
			$B$: "1 là số nguyên tố".
		\end{vd}
		\switchcolumn
		\begin{luyentap}
			Nêu ví dụ về một mệnh đề đúng và một mệnh đề sai.
		\end{luyentap}
	\end{paracol}
%%%==============HetVidu1==============%%%
\subsection{Mệnh đề chứa biến}
	\begin{hd}
		Xét câu " $n$ chia hết cho 3 " với $n$ là số tự nhiên.
		\begin{enumerate}[a)]
			\item  Ta có thể khẳng định được tính đúng sai của câu trên hay không?
			\item  Vối $n=21$ thì câu " 21 chia hết cho 3 " có phải là mệnh đề toán học hay không? Nếu là mệnh đề toán học thì mệnh đề đó đúng hay sai?
			\item  Vối $n=10$ thì câu " 10 chia hết cho 3 " có phải là mệnh đề toán học hay không? Nếu là mệnh đề toán học thì mệnh đề đó đúng hay sai?
		\end{enumerate}
	\end{hd}
	\begin{hd}
		Với mỗi mệnh đề chứa biến sau, tìm những giá trị của biến để nhận được một mệnh đề đúng và một mệnh đề sai.
		\begin{enumerate}[a)]
			\item  $P(x)$: " $x^2=2$ ";
			\item  $Q(x):$: $x^2+1>0$ ",
			\item  $R(n)$: " $n+2$ chia hết cho 3 " ($n$ là số tự nhiên).
		\end{enumerate}
	\end{hd}
\subsection{Phủ định của một mệnh đề}
\subsection{Mệnh đề kéo theo}
\subsection{Mệnh đề đảo. hai mệnh đề tương đương}
\subsection{Kí hiệu $\forall$ và $\exists$}

\end{document}
